\documentclass[10pt,a4paper]{article}
\usepackage[utf8]{inputenc}
\usepackage[french]{babel}
\usepackage[T1]{fontenc}
\usepackage{amsmath}
\usepackage{amsfonts}
\usepackage{amssymb}
\usepackage{svg}
\usepackage{multicol}
\usepackage{textcomp}

\title{Formulaire de physique}

\begin{document}
\section{Physique du point Matériel}

\subsection{Noms de variables}

\begin{tabular}{|c|c|c|}
\hline 
Grandeur & Qu'est-cequec'estquoi & Unités \\ 
\hline 
$M$ & Masse & $[kg]$ \\
\hline
$x(t)$ & Position en fonction du temps & $[m]$ \\ 
\hline
$v(t) = \dot{x}(t)$ & Vitesse & $[\frac{m}{s}]$ \\
\hline

$a(t) = \ddot{x}(t)$ & Acceleration & $[\frac{m}{s^2}]$ \\

\hline
$F_{cause}$ & Force associée à une cause & $[N] = [\frac{kg*m}{s^2}]$ \\
\hline
$p(t) = mv(t)$ & Quantité de mouvement & $[\frac{kg*m}{s}]$ \\
\hline
$k$ & Constante d'élasticité du ressort & $[\frac{kg}{s^2}]$\\
\hline
$E_{cin}$ & Energie cinétique du point & $[J] = [\frac{kg*m^2}{s^2}$ \\
\hline
\end{tabular}

\subsection{Formules}


$\sum F_{ext} = ma$, $W_F = \int\limits_a^b F dx$

\subsubsection{Expression de forces}

\begin{tabular}{c|c|c}
Nom & Raccourci usuel & Expression \\
\hline
Force pesante & $F_p$ & $mg$, $g$ accélération terrestre \\
Force de rappel du ressort & $F_k$ & $-k\Delta x$, k elasticité du ressort\\
Force normale & $N$ & déterminer par Newton \\
Force de Tension & $T$ & déterminer par Newton \\
Force centrifuge (fictive) & $-$ & $-m\vec\omega \times (\vec\omega \times \vec R)$ \\
Force de coriolis (fictive) & $-$ & $-2m\omega \times \vec{v'}$ \\
Force liée à $\dot{\vec{\omega}}$ & $-$ & $\dot{\vec{\omega}} \times \vec R$ \\
Force de gravitation & $F_{12}$ & $-G\frac{m_1 \cdot m_2}{d^2}\vec{u_{12}}$ \\
\end{tabular}

\subsubsection{Expression d'Énergies}

\begin{tabular}{c|c|c}
Nom & Raccourci usuel & Expression \\
\hline
Énergie potentielle de gravité & $E_{g}$ & $mgh$ \\
Énergie potentielle du ressort & $E_k$ & $\frac{1}{2}kx^2$ \\
Énergie cinétique & $E_{cin}$ & $\frac{1}{2}mv^2$ \\
\end{tabular}

La variation de l'énergie cinétique est égale à la somme des travaux de toutes les forces.

\subsubsection{Solutions d'equations}

\begin{tabular}{c|c|c}
Mouvement & Equadiff & Solutions \\
\hline
Oscillateur harmonique & $ddot{x} + \omega^2x = 0$ & $Acos(\omega_0t) + Bsin(\omega_0t) = Csin(\omega_0t+D)$ \\
Oscillateur amorti & $\ddot{x} + 2\gamma\dot{x}+\omega^2_0x = 0$ & - \\
Oscillateur amorti forcé & $\ddot{x} + 2 \gamma \dot{x} + \omega^2_0 x = \alpha_0sin(\omega t)$ & - \\
\end{tabular}

Avec:

$\omega_0 = \sqrt{\frac{k}{m}}$,$T = \frac{2\pi}{\omega_0}$,$\nu = \frac{1}{T}$,$\gamma = \frac{b}{2m}$,$A = x_0$,$B = \frac{v_0}{\omega_0}$, $C^2 = x^2_0 + (\frac{v_0}{\omega_0})^2$, $tg(D) = \omega_0 \frac{x_0}{v_0}$,$\alpha_0 = \frac{f}{m}$

\pagebreak

\subsection{Systemes de coordonnées}

Loi de poisson :

\begin{center}
$\dot{\hat{\mathbf{e}}}_i = \omega \times \hat{\mathbf{e}}_i$ , ou $\omega$ est la vitesse angulaire du référentiel.
\end{center}



\subsubsection{Coordonnées Sphériques}

$$\omega = \dot\theta \hat e_z + \dot\phi \hat e_\theta$$

$$\sum F_{ext} = ma$$

\subsection{Systemes de coordonnées}
\subsubsection{Coordonnées Sphériques}

\begin{multicols}{3}

\begin{align*}
x & = \rho \sin \phi \cos \theta \\
y & = \rho \sin \phi \sin \theta \\
z & = \rho \cos \phi
\end{align*}

\columnbreak

\includesvg[width = 0.25\textwidth]{sphere}

\columnbreak


\begin{align*}
\rho   &= \sqrt{x^2+y^2+z^2}\\
\phi &= \arccos(z/\rho)\\
\theta &= \begin{cases}\arccos\frac{x}{\sqrt{x^2+y^2}} & \mathrm{si}\ y\geq{0} \\ 2\pi-\arccos\frac x{\sqrt{x^2+y^2}} & \mathrm{si}\ y < 0\end{cases}\\
\theta &= \arctan(y/x)
\end{align*}

\end{multicols}

Vecteur rayon et dérivées:

\begin{align*}
\vec{r} &= \rho \hat{e}_{\rho} \\
\dot{\vec{r}} &= \dot{\rho}\hat{e}_{\theta} + \rho\dot{\phi}\sin\theta\hat{e}_{\phi} = \vec{v} \\
\begin{split}
\ddot{\vec{r}} &= \left( \ddot{\rho} - \rho\dot\theta^2 - \rho\dot\phi^2\sin^2\theta \right)\mathbf{\hat{e}_\rho} \\
&+ \left( \rho\ddot\theta + 2\dot{\rho}\,\dot\theta - \rho\dot\phi^2\sin\theta\cos\theta \right) \mathbf{\hat{e}_\theta}\\
&+ \left( \rho\ddot\phi\,\sin\theta + 2\dot{\rho}\,\dot\phi\,\sin\theta + 2 \rho\dot\theta\,\dot\phi\,\cos\theta \right) \mathbf{\hat{e}_\phi} = \vec{a}
\end{split}
\end{align*}

\subsubsection{Coordonnées Cylindriques}


$$\omega = \dot\theta \hat e_z$$

\begin{multicols}{3}


\begin{align*}
x &= r  \cos\theta \\
y &= r  \sin\theta \\
z &= z
\end{align*}

\columnbreak

\includesvg[width = 0.3\textwidth]{cylindre}

\columnbreak

\begin{align*}
r & = \sqrt{x^2 + y^2}\\
\theta & = \arctan\left({y \over x}\right) \\
z & = z
\end{align*}

\end{multicols}

Vecteur rayon et dérivées:
\begin{align*}
\vec{r} &= r \hat e_r + z\hat e_z \\
\dot{\vec{r}} &= \dot r \hat e_r + r \dot\theta\hat e_\theta + \dot z \hat e_z = \vec{v} \\
\ddot{\vec{r}} &= \left(\ddot r - r\dot\theta^2 \right)\hat e_r 
+ \left( r\ddot\theta + 2\dot r \dot \theta \right) \hat e_\theta + \ddot z \hat e_z 
\end{align*}


\section{Physique du corps Solide}

\subsection{Noms de variables}

\begin{tabular}{|c|c|c|}
\hline 
Grandeur & Qu'est-cequec'estquoi & Unités \\ 
\hline 
$I$ & Repartition de la masse pondérée par la distance au carré & $[kg*m^2]$ \\ 
\hline
$\omega$ & Vitesse angulaire , vecteur parralèle à l'axe de rotation. & $[\frac{rad}{s}]$ \\ 
\hline
$L$ & Moment cinétique & $[\frac{kg*m^2}{s}]$ \\
\hline
$\alpha$ & Acceleration Angulaire & $[\frac{rad}{s^2}]$ \\
\hline
\end{tabular}

\subsection{Formules}
$M_F = \vec F \times \vec R$,
$\sum \vec{M_{ext}} = I\vec{\alpha}$ ,
$\frac{d\vec{L}}{dt} = \sum \vec{M_{ext}}$,
$\vec{v} = \vec{\omega} \times \vec{r}$,
$\vec{a_t} = \vec{\alpha} \times \vec{r}$,
$I = \sum \limits_M d^2 dm$,
$E_{cin} = \frac{1}{2}I\omega^2$

\section{Systeme de points materiels}

Equation du mouvement relatif :

\begin{center}
$\vec{F}_{2 \leftarrow 1} = \mu \ddot\vec{r}$ où $\mu = \frac{m_1 m_2}{m_1 + m_2}$ la masse réduite.
\end{center}
\end{document}